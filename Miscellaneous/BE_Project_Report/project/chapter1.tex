\chapter{Introduction}
 \section{Problem Statement}

To develop a tool for detecting plagiarism in software source code using
the machine learning algorithms.
 \section{Scope of the Project}

 \begin{itemize}
\item For the first level of implementation , tool will be on a local machine for
checking plagiarism.
\item The source and target program should be in Java.
 \end{itemize}
 \section{Current Scenario}
\begin{table}[]
\centering
\begin{tabular}{|l|l|l|l|l|}
\hline
          & MOSS             & SHERLOCK                      & JPLAG                & CODEMATCH                     \\ \hline
Cost      & Free             & Open Source                   & Free                 & It is a Commercial Tool       \\ \hline
Safety    & Sign-in Required & Executes at Local machine     & Sign-in Required     & Executes at Local machine     \\ \hline
Service   & Internet         & Standalone                    & Web-Service          & Standalone                    \\ \hline
Algorithm & Winowing         & Token Matching                & Greedy String Tiling & String Matching               \\ \hline
Speed     & Fast             & More files requires more time & Fast                 & More files requires more time \\ \hline
\end{tabular}
\end{table}
 \section{Need for the Proposed System}
The Role of Plagiarism detection in Education, Role of Plagiarism Detection in Software,Industry especially copyright of software.
 \section{Summary of the Results / Task completed }
Study of different Research Papers,Thesis on Source Code Plagiarism.
Survey on different source code plagiarism tools.
Implementation upto Level 3 Plagiarism (Rule-based).
